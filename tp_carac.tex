\documentclass[a4paper,12pt]{article}


\usepackage[T1]{fontenc}
\usepackage[utf8]{inputenc}
\usepackage[frenchb]{babel}


\usepackage{graphicx}
\usepackage{xspace}


\title{Compte rendu de travaux pratiques\\ \small ou un meilleur titre}
\author{Vincent Denechaud, Olivier Maillet, Alice Odier, Félix Tora}
\date{Vendredi 24 janvier 2014}

\newcommand\ett{Everhart et Thornley\xspace}




\begin{document}

\maketitle

Voilà où on pourrait mettre l'introduction, sur les techniques MEB sans trop en faire.


\section{Rappel des différentes sources électroniques et leur détection}

\subsection{Les différentes sources électroniques du MEB}

Des électrons sont accélérés depuis la cathode vers l'échantillon que l'on souhaite étudier. Du fait de leur énergie cinétique, les électrons pénètrent dans la matière de l'échantillon en formant ce que l'on appelle une poire d'interaction (figure \ref{fig:poire_int}).

\begin{figure}
\centering
\includegraphics[width = 0.8 \textwidth]{images/poire_int.png}
\caption{Poire d'interaction : on distingue les différents type de rayonnements issus de l'interaction entre le faisceau électronique incident et l'échantillon.}
\label{fig:poire_int}
\end{figure}


En réponse à cette excitation, la matière revient à l'équilibre en émettant du rayonnement X ainsi que des électrons, 
de sorte que le cortège électronique des atomes composant l'échantillon possède une configuration d'équilibre énergétique. 
On peut classer les émissions électroniques selon différentes catégories.

\subsubsection*{Électrons rétro-diffusés}
Lorsque les électrons viennent interagir avec l'échantillon, ils sont, pour la plupart, rétro-diffusés de manière élastique.
Autrement dit, ces électrons interagissent avec les noyaux des atomes de l'échantillon de sorte à conserver leur énergie cinétique.
Ainsi, ces électrons ont une grande profondeur d'échappée et un grand libre parcours moyen dans l'échantillon.
De fait, ces électrons sont très sensibles à la composition chimique de l'échantillon, plus particulièrement, 
au numéro atomique Z des atomes constituant l'échantillon.
Les atomes les plus lourds (Z grand) vont ré-émettre plus d'électrons que les atomes plus légers. L'utilisation des électrons rétro-diffusés permet donc d'avoir un contraste chimique sur l'observation de l'échantillon.
Même s'ils ont une grande longueur de pénétration, ces électrons permettent une résolution spatiale (contraste topographique) de Xnm (figure \ref{fig:poire_int}).

\subsubsection*{Électrons secondaires}
En autre des électrons diffusés de manière quasi-élastique, une autre partie de ces derniers est diffusée de manière inélastique par l'échantillon.
Ce sont les électrons dits secondaires. 
Ces électrons cèdent donc très vite leur énergie cinétique et ont un faible libre parcours moyen ainsi qu'une faible profondeur de pénétration dans l'échantillon. 
Néanmoins, étant donné que ces derniers ressortent très vite de l'échantillon, la résolution spatiale obtenue 
après leur détection est bien meilleure (de l'ordre de $5$nm) que celle obtenue avec les électrons rétro-diffusés.
Les électrons secondaires seront donc très utiles pour réaliser un bon contraste topographique.
On notera cependant qu'il y a beaucoup moins d'électrons du type secondaire que rétro-diffusé. 
L'utilisation des électrons secondaires nécessite donc un temps de pose plus long pour réaliser un cliché.

\subsection{Les techniques de détection associées}

\subsubsection*{Détection des électrons rétrodiffusés}

\subsubsection*{Détection des électrons secondaires}

La détection des électrons rétro-diffusés se fait à l'aide du détecteur d'\ett. Il est composé d'une grille, d'un scintillateur et d'un photo-multiplicateur (figure \ref{fig:detect_ett}).

\begin{figure}
\centering
\includegraphics[width = 0.9 \textwidth]{images/detect_ett.png}
\caption{Détecteur d'\ett composé d'un collecteur, d'un scintillateur et d'un photo-multiplicateur.}
\label{fig:detect_ett}
\end{figure}
 
On polarise le collecteur en tension à $+200$ Volts, ce qui a pour rôle d'attirer les électrons secondaires. 
Cette grille joue aussi le rôle de "cage de Faraday", ce qui permet de ne pas attirer des électrons du faisceau incident. 
Les électrons secondaires possédant une faible énergie, ils sont ainsi sélectionnés par rapport aux électrons rétro-diffusés.
Il reste néanmoins les électrons rétro-diffusés avec l'angle solide correspondant à la direction du détecteur d'\ett dont l'on ne peut pas se passer.

Une fois les électrons collectés, ils sont accélérés vers un scintillateur polarisé à $10$kV qui les "transforme" en photons.
Ces photons sont ensuite conduits vers un photo-multiplicateur qui a pour but d'amplifier le signal final obtenu.

Il est possible d'utiliser le détecteur d'\ett  pour détecter les électrons rétro-diffusés.
En effet, en polarisant négativement le collecteur, les électrons secondaires émis à la surface de l'échantillon sont repoussés.
Ainsi, on ne détecte que les électrons rétro-diffusés possédant l'angle solide associé à la position du détecteur d'\ett, dit électrons rétro-diffusés à incidence rasante.
Une telle manipulation permet d'obtenir des informations complémentaires sur le contraste topographique.


\section{Echantillon éponge de Nickel}

L'observation de ce premier échantillon, une éponge de nickel, va nous permettre d'illustrer l'utilisation des différentes techniques présentées précédemment.

\subsection{Détecteur d'\ett}

Utilisant comme technique d'imagerie intiale les électrons secondaires recueillis par le détecteur d'\ett,
on accède à une première observation de l'échantillon présentée figure \ref{fig:ni_es}. Comme déjà évoqué,
cette technique a l'avantage d'offrir une grande profondeur de champ à l'image obtenue. Cette première
technique d'imagerie permet un contraste de profondeur de champ et donc de visualiser spatialement la forme
de l'échantillon.

\begin{figure}
\centering
\includegraphics[width = 0.7 \textwidth]{images/ni_es.png}
\caption{Observation de l'éponge de nickel utilisant les électrons secondaires recueillis par le détecteur d'\ett. L'échantillon est ici grossi 200 fois.}
\label{fig:ni_es}
\end{figure}

Le détecteur d'\ett permet aussi de récolter des électrons rétro-diffusés. Ces électrons rétro-diffusés sont
collectés dans un angle solide bien précis, ainsi l'image obtenue à l'aide de cette technique présente une
certaine profondeur de champ mais aussi des effets d'ombre et de lumière. Ces effets sont dus à l'angle
d'incidence des électrons collectés. L'image obtenue à l'aide de cette technique est présentée figure
\ref{fig:ni_er_rasant}.

\begin{figure}
\centering
\includegraphics[width = 0.7 \textwidth]{images/ni_er_rasant.png}
\caption{Observation de l'éponge de nickel utilisant les électrons rétro-diffusés recueillis par le détecteur d'\ett. L'échantillon est ici grossi 200 fois.}
\label{fig:ni_er_rasant}
\end{figure}


\subsection{Détecteur à semi-conducteur}

L'utilisation du détecteur à semi-conducteur permet de collecter un grand nombre d'électrons rétro-diffusés
et d'obtenir deux types d'imagerie. L'utilisation du mode $A+B$ permet d'obtenir une image avec un contraste
en composition chimique. L'image de l'éponge de nickel obtenue avec cette technique est présentée à la figure
\ref{fig::ni_er_apb}. Sur cette image on voit que la profondeur de champ du cliché est très faible. Cependant
le contraste en composition chimique permet d'obtenir d'autres informations tout aussi importantes.

On peut voir sur l'échantillon certaines excroissances qui apparaissent plus foncées. Cette image permet d'observer
la présence d'impuretés sur le matériau, tout du moins de parties faites d'éléments différents.


\begin{figure}
\centering
\includegraphics[width = 0.7 \textwidth]{images/ni_er_apb.png}
\caption{Observation de l'éponge de nickel utilisant les électrons rétro-diffusés recueillis par le détecteur à semi-conducteur. L'échantillon est ici grossi 200 fois.}
\label{fig:ni_er_apb}
\end{figure}



Le détecteur à semi-conducteur permet également de réaliser un cliché différent en utilisant les mêmes électrons
rétro-diffusés. L'image réalisée avec le mode $A-B$ est présentée sur la figure \ref{fig:ni_er_amb}. Sur ce cliché
le contraste est topographique.

\begin{figure}
\centering
\includegraphics[width = 0.7 \textwidth]{images/ni_er_amb.png}
\caption{Avec une jolie légende en prime}
\label{fig:ni_er_amb}
\end{figure}
 

\section{On pourrait mettre une deuxième partie ici}

Sur un autre échantillon.

\vspace{5cm}

Et imaginer d'autres parties, des sous parties et des jolies images comme sur la figure \ref{fig:ni_er_amb}.

\begin{figure}
\centering
\includegraphics[width = 0.7 \textwidth]{images/ni_er_amb.png}
\caption{Avec une jolie légende en prime}
\label{fig:ni_er_amb}
\end{figure}

\section*{Conclusion}

Et puis tout à la fin on pourrait mettre une petite conclusion aux petits oignons.


\end{document}
